\section{WIMSR}
\label{sWIMSR}

\hypertarget{sWIMSRhy}{The}
WIMSR\index{WIMSR|textbf} module is used to prepare libraries for the
reactor-physics code WIMS\cite{WIMS}.\index{WIMS}  WIMS stands for
``Winfrith improved multigroup scheme;''\index{AEE/Winfrith}
it  has been developed through its various versions
at the UK laboratory AEE/Winfrith.  WIMS-E was the Winfrith
version {\it circa} 1990, and it was distributed commercially.
WIMS-D is an older version that is freely available through various
distribution centers; therefore, it is very popular all around the world.
\index{WIMS!WIMS-D}
\index{WIMS!WIMS-E}

WIMS uses collision-probability methods\index{collision-probability method}
for computing fluxes in reactor pin cells and more complicated
geometrical arrangements.  Therefore, it requires transport,
fission, and capture cross sections, a transfer matrix for epithermal
neutrons, fission-source information ($\overline{\nu}$ and $\chi$),
and a bound-atom scattering matrix for thermal neutrons.  Self-shielded
\index{self-shielding}
cross sections are obtained using equivalence theory\index{equivalence theory}
from tabulated resonance integrals\index{resonance integrals}
with intermediate resonance corrections.
\index{intermediate resonance effects}
The resonance integrals can be obtained from the
self-shielded cross sections produced by
\hyperlink{sGROUPRhy}{GROUPR}\index{GROUPR}, and the
intermediate-resonance $\lambda$ values by group can
be computed using the NJOY flux calculator\index{flux calculator}.
WIMS libraries normally use a standard 69-group structure with
14 fast groups, 13 resonance groups, and 42 thermal groups.
\index{69-group structure}

This chapter describes the WIMSR module in NJOY2016.0.

\subsection{Resonance Integrals}
\label{ssWIMSR_RI}

WIMS computes the self-shielded\index{self-shielding} cross sections
for a wide range of mixtures and fuel geometries using equivalence theory.
The \hyperlink{sGROUPRhy}{GROUPR} chapter of this
report describes the narrow-resonance
(NR) version of equivalence theory; that is, all systems with
the same value for the ``sigma-zero,"\index{$\sigma_0$}

\begin{equation}
   \sigma_{0i}=\frac{1}{n_i}\left\{\sum_{j\ne i} n_j\sigma_{tj}
       +\sigma_e\right\}\;,
\end{equation}

\noindent
in a group have the same self-shielded cross section in that
group.  Here, $n_j$ is the number density for material $j$
with cross section $\sigma_{tj}$, and $\sigma_e$ is the escape
cross section (which takes care of the geometry of the fuel).

However, in the near epithermal range ({\it e.g.,} 4-100 eV),
some resonances are too wide for the NR approximation to apply
well.\index{narrow resonance approximation}  For these resonances,
the effect of a moderator material is reduced, because collisions
with the moderator do not always result in enough energy loss
to remove the neutron from the resonance.  For this reason, WIMS
uses an intermediate-resonance (IR) extension\index{intermediate
resonance effects} to equivalence theory in which the background
cross section is taken to be the following:

\begin{equation}
   \sigma_{Pi}=\frac{1}{n_i}\left\{\sum_{j} n_j
       \lambda_j\sigma_{pj} +\sigma_e\right\}\;,
\end{equation}
\vspace{0.5 pt}

\noindent
where the $\lambda$ factors\index{IR $\lambda$} are numbers between
zero and one.  Note that $\sigma_p$, the potential scattering cross
section, is used here, and that the sum is now over all materials.
The basic concept is the same:  all systems with the same
value of the IR ``sigma-P'' for a group will have the same
self-shielded cross sections for that group.

WIMS takes the additional step of expressing the self-shielding
data in terms of ``resonance integrals,''
\index{resonance integrals}
instead of using the self-shielded cross sections produced by
\hyperlink{sGROUPRhy}{GROUPR}.  That is

\begin{equation}
   \sigma_x(\sigma_0)=\frac{\sigma_P I_x(\sigma_P)}
      {\sigma_P - I_a(\sigma_P)}\;,
\end{equation}
\vspace{0.5 pt}

\noindent
and

\begin{equation}
   I_x(\sigma_P)=\frac{\sigma_P\sigma_x(\sigma_0)}
     {\sigma_P + \sigma_a(\sigma_0)}\;,
\end{equation}
\vspace{0.5 pt}

\noindent
where $x$ denotes the reaction, either ``a'' for absorption or
``nf'' for nu*fission, and $I_x$ is the corresponding resonance integral.

In order to clarify the meaning of this pair of equations, consider
a homogeneous mixture of $^{238}$U and hydrogen with concentrations
such that there are 50 barns of hydrogen scattering per atom of
uranium.  The \hyperlink{sGROUPRhy}{GROUPR} flux
calculator can be used to solve for the
flux in this mixture, and
\hyperlink{sGROUPRhy}{GROUPR} can then calculate the corresponding
absorption cross section for $^{238}$U.  Assuming that $\lambda=0.1$
and $\sigma_p=10$ for the uranium, the numbers being appropriate
for WIMS group 25, we get $\sigma_P=51$.  This value of $\sigma_p$
goes into the \cword{sigz} array in the resonance-integral block on
the WIMS library, and the corresponding $I_a$ goes into the
\cword{resa} array.

At some later time, a WIMS user runs a problem for a homogeneous
mixture of $^{238}$U and hydrogen that matches these specifications.
WIMS will compute a value of $\sigma_P$ of 51.0, interpolate
in the table of resonance integrals, and compute a new absorption
cross section that is exactly equal to the accurate computed result
from the original \hyperlink{sGROUPRhy}{GROUPR} flux-calculator run.

This argument can be extended to more complex systems.  For example,
the assembly calculated using the flux calculator could represent
an enriched uranium-oxide fuel pin of a size typical of a user's
reactor system with a water moderator.  The computed absorption
cross section is converted to a resonance integral and stored with
the computed value of $\sigma_P$.  In any later calculation that
happens to mimic the same composition and geometry, WIMS will
return the accurate calculated absorption cross section.  Equivalence
theory\index{equivalence theory}, with all its approximations, is only
used to interpolate and extrapolate around these calculated values.
This is a powerful approach, because it allows a user to optimize the
library in order to obtain very accurate results for a limited range
of systems without having to modify the methods used in the
lattice-physics code.  Unfortunately, the present version of WIMSR
does not allow you to enter $\sigma_P$ directly; it computes it
from input data that only consider one material at a time.  A future
version may include the more general capabilities described in
this paragraph.

Let us call the homogeneous uranium-hydrogen case discussed above
``case 0.''  Now, consider a homogeneous mixture of $^{238}$U, oxygen,
and hydrogen.  Ratio the number densities to the uranium density such
that there is 1 barn/atom of oxygen scattering and 50 barn/atom of hydrogen
scattering.    Carry out an accurate flux calculation for the mixture, and
call the result ``case 1.''  Also do an accurate flux calculation with only
hydrogen,  but at a density corresponding to 51 barns/atom. Call this result
``case 2.''  The IR lambda value for oxygen is then given by

\begin{equation}
\lambda = \frac{\sigma_a(1)-\sigma_a(0)}{\sigma_a(2)-\sigma_a(0)}\;.
\end{equation}
\vspace{0.5 pt}

\noindent
Note that $\lambda$ will be 1 if the oxygen and hydrogen have
exactly the same effect on the absorption cross section.  In practice,
$\lambda=.91$ for WIMS group 27 (which contains the large 6.7 eV
resonance of $^{238}$U), and $\lambda=1$ for all the other resonance
groups.  That is, all the resonances above the 6.7 eV resonance are
effectively narrow with respect to oxygen scattering.

Similarly, do a flux-calculator solution for a homogeneous mixture
of $^{238}$U combined with 1 barn/atom of $^{235}$U and 50 barn/atom
of hydrogen.  Call the result ``case 3.''  Now, the lambda value
for $^{235}$U is given by

\begin{equation}
\lambda = \frac{\sigma_a(3)-\sigma_a(0)}{\sigma_a(2)-\sigma_a(0)}\;.
\end{equation}
\vspace{0.5 pt}

\noindent
The actual value obtained for WIMS group 27 is .035.  Group 26 gives
0.50, and group 25 gives 0.09. An examination
of the flux-calculator equations in the
\hyperlink{sGROUPRhy}{GROUPR} chapter of this
manual shows that the effect of the ``admixed'' moderator term
depends only on its atomic mass (through the $\alpha$ value); therefore,
the IR $\lambda$ values will be the same for all uranium isotopes
(and the same values should work for all the actinides).  This
conclusion neglects the small effects of absorption in the admixed
isotope on the intraresonance flux for one resonance.

This process can be continued for additional admixed materials from
each important range of atomic mass.  The result is the table of
$\lambda_{gi}$ values needed as WIMSR input.

What are the implications of this discussion?  Foremost is the observation
that the lambda values for the isotopes are a function of the composition
of the mixture that was used for the base  calculation.  To make the
effect of this clear, let us consider two different types of cells:
\begin{enumerate}
\begin{singlespace}
\item a homogeneous mixture of $^{238}$U and hydrogen, and

\item a homogeneous mixture of $^{235}$U and hydrogen.
\end{singlespace}
\end{enumerate}
\vspace{0.5 pt}

A look at the pointwise cross sections in group 27 shows very different
pictures for the two uranium isotopes.  The $^{238}$U cross section has one
large, fairly wide resonance at 6.7 eV, and the $^{235}$U cross section has
several narrower resonances scattered across the group.  If the
lambda values are computed for these two different situations,  the
results in Table~\ref{lams} are obtained.

\begin{table}[b]
\begin{center}
\caption[Sample Intermediate Resonance $\lambda$ Values]{IR $\lambda$ Values
 for Several Resonance Groups and Two Different Reactor Systems}
\setlength{\extrarowheight}{1pt}
\begin{tabular}{c|cc|cc} \hline
 WIMS  &  $\lambda$(U)   &  $\lambda$(O)  & $\lambda$(U)  &  $\lambda$(O) \\
 group & $^{238}$U@50b   & $^{238}$U@50b   &  $^{235}$U@200b
  & $^{235}$U@200b  \\ \hline
 27  &  .035  &  .91  &  0.20  &  1.00 \\
 26  &  .50   &  1.00 & .38  &  1.00 \\
 25 &  .092  &  1.00  & .44  &  1.00  \\
 24  &  .090  &  1.00  &  .55  &  1.00  \\
 23  &  .29  &  1.00  &  .46  &  1.00 \\ \hline
\end{tabular}
\label{lams}
\end{center}
\end{table}

\newpage
It is clear that the energy dependence of the two lambda sets is quite
different.  This is because of the difference in the resonance structure
between $^{238}$U and $^{235}$U.  Clearly, the one resonance in group 27
in $^{238}$U
is effectively wider than the group of resonances in group 27 for $^{235}$U.
Group 26 has essentially no resonance character for $^{238}$U, which
reverses the sense of the difference. In groups 24 and 25, the $^{235}$U
resonances become more narrow, while the $^{238}$U resonances stay fairly
wide.  Finally, in group 23, the $^{238}$U resonances begin to get narrower.

These results imply that completely different sets of lambda
values should be used for different fuel/moderator systems, such as
$^{238}$U/water, $^{235}$U/water, or $^{238}$U/graphite.  In practice, this is
rarely done.

The remaining question is, ``How should the self-shielded cross
sections for the minor isotopes be calculated?''  Formally, the
best approach using NJOY would be to first do an accurate flux
calculation for pure $^{238}$U mixed with hydrogen (or to be
really accurate, a typical reactor cell containing pure $^{238}$U
oxide), and to save the resulting flux on a scratch file.  This
flux would then be used as input for the $^{235}$U calculation.  See
the \hyperlink{sGROUPRhy}{GROUPR} chapter for details.  This
approach takes care of all
the complexities of resonance-resonance interference, the drop
in the average flux across the group caused by accumulated
$^{238}$U absorptions, and so on.  In practice, this is rarely
done.  Since the self-shielding effects in the minor actinides are
much smaller than those in the main absorber, it is usually sufficient
to do a simple NR calculation for the minor actinides and to
convert them into WIMS resonance integrals with the normal lambda
values for heavy isotopes.

\subsection{Cross Sections}
\label{ssWIMSR_xs}

The first part of the WIMS cross section data contains $\sigma_p$
for the resonance groups (15-27 in the normal 69-group structure),
the scattering power per unit lethargy for the resonance groups,
the transport cross section for the non-thermal groups (1-27 normally),
the absorption cross section for the non-thermal groups, an obsolete
quantity for the resonance groups (set to zero), and the
intermediate-resonance $\lambda$ values for the resonance groups.
For WIMS-E, the (n,2n) cross section is added between the
slowing-down power and the transport cross section.

The $\sigma_p$ value is assumed to be constant (see \cword{sigp} in the
user input instructions).  It must be obtained by finding the scattering
length $a$ in the ENDF file and computing $\sigma_p=4\pi a^2$.  The
scattering power per unit lethargy is $\xi\sigma_s/\tau$, where $\xi$
is the log energy loss parameter given as MT=252 on the GENDF file,
$\sigma_s$ is the elastic scattering cross section (MT=2), and $\tau$
is the lethargy width for the group, which can be calculated from the
group structure given in MF=1/MT=451 on the GENDF file.  The (n,2n)
cross section is obtained from MF=3/MT=16 on the GENDF file.  The
absorption cross section is computed by adding up the fission
cross section (MT=18) and all the cross sections given with
MT=102-150.  The (n,2n) cross section (MT=16) is then subtracted
from the sum.  Finally, the $\lambda$ values are obtained from the
user's input.  See Section~\ref{ssWIMSR_RI} for more details on these
intermediate-resonance corrections.

The next part of the WIMS data file contains the fission neutron
production cross section $\overline{\nu}\sigma_f$ and the fission
cross section $\sigma_f$ for the non-thermal groups (1-27 normally).
The cross section is always obtained from MT=18 on the GENDF
file, but there are several complications involved in getting
$\overline{\nu}$.

A shortcut for obtaining the fission data is to run MFD=3/MTD=452
and MFD=5/MTD=452 in \hyperlink{sGROUPRhy}{GROUPR}.  This
approach ignores the energy dependence
of fission neutron emission at high energies and the effects of
delayed neutrons on the fission spectrum.  If these options are
used in GROUPR, it is important not to use the other options
described below at the same time.  When WIMSR finds a section on
the GENDF file with MF=3/MT=452, it can read in $\overline{\nu}\sigma_f$
directly.

A better approach to fission in \hyperlink{sGROUPRhy}{GROUPR}
is to prepare a full fission
matrix\index{fission matrix} for MT=18, or to prepare matrices
for all the partial fission reactions, MT=19, 20, 21, and 38.
The latter is the recommended approach for evaluations with both
MT=18 and MT=19 given in File 5.  See the
\hyperlink{sGROUPRhy}{GROUPR} chapter for more
details.  WIMSR reads in the data given in MF=6/MT=18, or in
MF=6/MT=19,20,..., and sums over all secondary-energy groups
to obtain the prompt part of $\overline{\nu}\sigma_f$.  It adds
in the delayed part of $\overline{\nu}\sigma_f$ from MF=3/MT=455.
If the input GENDF file contains both MT=18 and partial fission matrices,
a diagnostic message will be printed, and the partial-fission
representation will be used.

The next section of the WIMS data file contains the non-thermal P$_0$
scattering matrix for incident-energy groups in the non-thermal range
(1-27 normally).  This matrix is loaded by summing over all of the
reactions found on the GENDF tape except the thermal reactions
\cword{mti} and \cword{mtc}.  If requested, this matrix is transport
corrected by subtracting the sum over secondary-energy groups of the
P$_1$ matrix for each primary group.  When the individual reactions
are read, they are loaded into ``full'' matrix (typically 69$\times$69).
At the same time, a record is kept of the lowest and highest secondary
groups found for each primary group.  These limits are then used to
pack the scattering matrix into a more compact form.

The scattering matrix is not actively self-shielded in WIMS, but WIMSR
allows the user to request that the elastic component be evaluated at
some reference $\sigma_0$ value different from infinity.  This option
can be useful for the major fertile component of reactor fuel, that is,
for $^{238}$U in pins of a uranium system, or for $^{232}$Th for fuel
in a Thorium/$^{233}$U system.

Because the thermal scattering matrix depends on temperature,
the next component of the WIMSR data contains the \cword{ntemp}
versions of the basic cross sections and the P$_0$ scattering matrix
for the thermal groups (28-69 normally).  The cross sections
included are transport, absorption, nu*fission, and fission.
The transport positions contain the sum of the thermal inelastic
cross section obtained by summing up the P$_0$ matrix (MF=6/MT=MTI),
the thermal elastic cross section from the diagonal elements of
the P$_0$ matrix (MF=6/MT=MTC), if present, and the absorption cross
section.  If separate P$_1$ matrices are not given for this material,
the P$_1$ cross section obtained by summing the P$_1$ matrices over
secondary groups for each primary group is subtracted.  The matrix
data are read from sections on the GENDF file with MF=6/MT=MTI and
MF=6/MT=MTC (if present).  As for the temperature-independent
matrices, they can be transport-corrected by subtracting the
sum over secondary groups of the P$_1$ matrix for each primary group
from the self-scatter position.  Also, minimum and maximum limits
on the secondary group are determined for each primary group, and the
matrix is compacted for efficiency.

The next part of the WIMS data file contains the resonance data,
which were discussed in Section~\ref{ssWIMSR_RI}.  In some
cases, these resonance data are followed by a fission spectrum
block.  The complications of obtaining the fission spectrum are the
same as those described above for obtaining the fission neutron
production cross section, $\overline{\nu}\sigma_f$.  If the short-cut
option was used in \hyperlink{sGROUPRhy}{GROUPR}, the
fission spectrum can be read directly
from MF=5/MT=452 on the GENDF file.  The sum over groups is also
accumulated in \cword{cnorm} in order to allow the final spectrum
to be normalized accurately.  The shortcut approach neglects the
effects of delayed neutron emission on the fission spectrum.

If fission matrices are available (either MT=18, or MT=19+20+21...),
the prompt part of the fission spectrum is obtained by summing
$\sigma_{gg'}\phi_g$ over all primary groups, $g$.  These numbers are
also summed into \cword{cnorm} for use later in normalizing $\chi$.
The delayed part is obtained as

\begin{equation}
 \left\{ \sum_g \nu_d \sigma_{fg} \phi_g \right\} \chi_{dg'} \;,
\end{equation}

\noindent
which also contributes to the normalization.  Note that the energy
dependence of the fission matrix is factored into the final $\chi$ in
proportion to the weighting flux used in
\hyperlink{sGROUPRhy}{GROUPR} to prepare the WIMSR
input file.  For thermal reactor problems, it is easy to provide a
good estimate for this weighting flux.

The final block on the WIMS data tape is optional.  If present,
it contains P$_1$ scattering matrices for each temperature.  These
matrices are defined over the entire group range (normally 1-69),
and they contain both the temperature-independent and
temperature-dependent reactions in each matrix.  The methods
used to build up these matrices are parallel to those discussed
above for the P$_0$ matrices.  Note that if P$_1$ matrices are given,
the transport corrections are \underline{not} included in the transport
cross sections or P$_0$ matrices.


\subsection{Burn Data}
\label{ssWIMSR_burn}

WIMS uses a simplified burn model for tracking the production and
depletion of actinides and fission products, and the chains used
are hard-wired into the code.  WIMSR provides a method to enter
new fission-yield data into the WIMS library format, but it has not
been used or tested very much so far.

\subsection{User Input}
\label{ssWIMSR_inp}

The following user input specifications were copied from the comment
cards at the beginning of the WIMSR source.  It is always a good idea
to check the comment cards in the current version to see if there
have been any changes.
\index{WIMSR!WIMSR input}
\index{input!WIMSR}

\small
\begin{ccode}

   !--------------------------------------------------------------------
   !
   ! Format multigroup cross sections from groupr for WIMS.
   !
   !---input specifications (free format)----
   !
   ! card 1
   !    ngendf  unit for input gendf tape
   !    nout    unit for output wims tape
   !
   ! card 2
   !    iprint  print option
   !               0=minimum (default)
   !               1=regular
   !               2=1+intermediate results
   !    iverw   wims version
   !               4=wims-d (default)
   !               5=wims-e
   !    igroup  group option
   !               0=69 groups (default)
   !               9=user's choice
   !
   ! card 2a  (igroup.eq.9 only)
   !    ngnd    number of groups
   !    nfg     number of fast groups
   !    nrg     number of resonance groups
   !    igref   reference group (default is last fast group)
   !
   ! card 3
   !    mat     endf mat number of the material to be processed
   !    nfid    not used
   !    rdfid   identification of material for the wims library
   !    iburn   burnup data option
   !              -1=suppress printout of burnup data
   !               0=no burnup data is provided (default)
   !               1=burnup data is provided in cards 5 and 6
   !
   ! card  4
   !    ntemp   no. of temperatures to process for this material
   !            in the thermal energy range
   !               (0=all found on input tape)
   !    nsigz   no. of sigma zeroes to process for this material
   !               (0=all found on input tape)
   !    sgref   reference sigma zero
   !               (.ge. 1.e10 to select all cross sect. at inf.dil.*
   !                           but fully shielded elastic x-sect,
   !                .lt. 1.e10 to select all x-sect at inf.dil.
   !                =sig0      from the list on groupr input to
   !                           select all x-sect. at that sig0)
   !    ires    resonance absorber indicator
   !               0=no resonance tables
   !              >0=ires temperatures processed
   !    sigp    potential cross section from endf.
   !               (if zero, replace by the elastic cross section)
   !    mti     thermal inelastic mt (default=0=none)
   !    mtc     thermal elastic mt (default=0=none)
   !    ip1opt  include p1 matrices
   !               0=yes
   !               1=no, correct p0 ingroups (default)
   !    inorf   resonance fission (if found)
   !               0=include resonance fission (default)
   !               1=do not include
   !    isof    fission spectrum
   !               0=do not include fission spectrum (default)
   !               1=include fission spectrum
   !    ifprod  fission product flag
   !               0=not a fission product (default)
   !               1=fission product, no resonance tables
   !               2=fission product, resonance tables
   !    jp1     transport correction neutron current spectrum flag
   !               0=use p1-flux for transport correction (default)
   !              >0=read in jp1 values of the neutron current
   !                   spectrum from input
   !
   !   the following cards 5 and 6 are for iburn gt 0 only
   ! card  5
   !    ntis    no. of time-dependent isotopes
   !            for burnable materials ntis=2
   !            for fissile materials ntis>2 when fission product
   !            yields are given.
   !    efiss   energy released per fission
   !
   ! card  6a
   !    identa  ident of capture product isotope
   !    yield   yield of product identa from capture
   !
   ! card  6b
   !    identa  ident of decay product isotope (zero if stable)
   !    lambda  decay constant (s-1)
   !
   ! card  6c  (repeated ntis-2 times, if necessary)
   !    identa  ident of fission product isotope
   !    yield   fission yield of identa from burnup of mat
   !
   ! card  7
   !    lambda  resonance-group goldstein lambdas (13 for
   !            default 69-group structure, nrg otherwise).
   !
   ! card  8    (only when jp1>0)
   !    p1flx   current spectrum (jp1 entries read, the rest are
   !            set with the default p1-flux calculated by njoy).
   !--------------------------------------------------------------------

\end{ccode}
\normalsize

The first card specifies the input and output unit numbers, as
is normal for NJOY modules.  \cword{ngendf} comes from a previous
\hyperlink{sGROUPRhy}{GROUPR} run, and it can be in
either binary or ASCII mode.
\cword{nout} is always in ASCII mode.

The options card allows the user to select how much detail will
be printed on the output listing (\cword{iprint}), whether the
output is intended for  WIMS-D or WIMS-E (\cword{iverw}), and how
many groups are desired.  Currently, the only difference between
WIMS-D and WIMS-E output is that some additional reaction cross
sections are included for the latter.  If the user selects some
group structure different from the standard 69-group structure,
an additional input card is required to give the number of groups
(\cword{ngnd}), the number of fast groups (\cword{nfg}, 14 for
69 groups), the number of resonance groups (\cword{nrg}, 13 for
69 groups), and the reference group used for normalizing the
flux (\cword{igref}, normally the low-energy group of the
fast groups).

Card 3 is required.  It gives the ENDF MAT number for
the materials to be processed.  If this MAT doesn't appear on
the GENDF tape, a fatal error message will be issued.  \cword{nfid}
will be the identification number for this material used on the
output WIMS library, and \cword{rdfid} will be the identification
number for the resonance data.  Formally, WIMS libraries allow for
data sets with more than one version of the resonance-integral
tabulation.  The last parameter on this card is \cword{iburn} to
flag whether burn data are included in the input stream.

Card 4 starts out with \cword{ntemp} and \cword{nsigz},
which define the size of the resonance-integral tables.  They
are normally both set to zero; the code then uses all of the
values computed by \hyperlink{sGROUPRhy}{GROUPR}.  The
reference sigma-zero value,
\cword{sgref}, is used for the elastic cross section and matrix,
because these quantities are not normally self-shielded by
WIMS.  Normally, \cword{1e10} is appropriate, but for the major
fissionable material in the reactor ({\it i.e.,} $^{238}$U or
$^{232}$Th), it may be better to use a realistic number like
\cword{sgref}=50.  WIMSR assumes that the potential scattering
cross section for the material is constant, but this constant
value is not available from the
\hyperlink{sGROUPRhy}{GROUPR} output.  The WIMSR user
will have to look in the ENDF-formatted evaluation for the
scattering length $a$, compute $\sigma_{p}=4\pi a^2$, and enter
the value as \cword{sigp}.  As shown in the following ENDF-formatted
file fragment, the scattering length (\cword{9.56630- 1} in this case)
is the second item on the fourth line in MF=2, MT=151.

\small
\begin{ccode}
 ...
 9.22350+ 4 2.33025+ 2          0          0          1          01395 2151
 9.22350+ 4 1.00000+ 0          0          1          2          01395 2151
 1.00000+ 0 8.20000+ 1          1          1          0          01395 2151
 3.50000+ 0 9.56630- 1          0          0          1          01395 2151
 2.33025+ 2 0.00000+ 0          0          0        780        1301395 2151
 ...
\end{ccode}
\normalsize

\vspace{1 pt}
The parameters \cword{mti} and \cword{mtc} select the thermal
inelastic and elastic data from the sections that might be available
on the GENDF tape.  Most materials have only free-gas scattering
available, and the appropriate values would be \cword{mti}=221
and \cword{mtc}=0.  The conventional values to use for reactor
moderator materials are given in Table~\ref{thermal}.
\index{thermal MT numbers}

\vspace{16 pt}
\begin{table}[b]
\caption[Conventional thermal material MT numbers used in WIMSR, GROUPR
 and THERMR]{Conventional values for thermal MT numbers (\cword{mti}
 and \cword{mte}) used in WIMSR, \hyperlink{sGROUPRhy}{GROUPR}, and
\hyperlink{sTHERMRhy}{THERMR}}
\label{thermal}
\setlength{\extrarowheight}{1pt}
\begin{center}
\begin{tabular}{lll}
Thermal Material & \cword{MTI} Value & \cword{MTC} Value \\ \hline
H in H$_2$O & 222 &   \\
D in D$_2$O & 228 &   \\
Be metal & 231 & 232 \\
Graphite & 229 & 230 \\
Benzine  &  227  &  \\
Zr in ZrH & 235 & 236 \\
H in ZrH  &  225 & 226 \\
Be(BeO)  &  233 & 234 \\
O(BeO)  & 237 & 238 \\
O(UO$_2$) & 239 & 240 \\
U(UO$_2$) & 241 & 242 \\
H in Polyethylene & 223 & 224 \\
Al metal & 243 & 244 \\
Fe metal & 245 & 246 \\ \hline
\end{tabular}
\end{center}
\end{table}

Continuing with Card 4, WIMSR allows P$_1$ scattering
to be treated in two ways.  If
\cword{ip1opt}=0, the P$_1$ matrix for the material is written
to the WIMS output file explicitly.   This option is normally
used only for major moderator materials, such as the components
of water.  The other option, \cword{ip1opt}=1, instructs the
code to use the P$_1$ data to transport-correct the P$_0$
elastic scattering matrix; that is, the ingroup elements of
the P$_0$ matrix are reduced by the sum over all outgoing
groups of the P$_1$ matrix for that ingoing group.

The \cword{inorf} parameter can be set to 1 to eliminate the
resonance-integral table for nu*fission from the WIMS output.
Some of the higher actinides are treated this way for some
WIMS libraries.  The \cword{isof} flag is set to 1 to tell
WIMSR to produce a fission spectrum.  This is usually done for
main fissile materials, such as $^{235}$U and $^{239}$Pu.  The
\cword{ifprod} flag is used to control whether resonance tables
are included for fission products.

WIMSR has some capability to format burn data for incorporation
into a WIMS library (see cards 5 and 6).  This part of the code
has not been used or tested very much.

The final card gives the intermediate-resonance $\lambda$
values for each of the resonance groups.  Methods for obtaining
these quantities with NJOY are outlined in Section~\ref{ssWIMSR_RI}.


\subsection{Coding Details}
\label{ssWIMSR_coding}

The main entry point for WIMSR is subroutine \cword{wimsr}
exported by module \cword{wimsr}\index{modules!wimsr@{\ty wimsr}}.
WIMSR starts by reserving the scratch files with unit numbers
from 10 through 14.  The next step is to read and echo the user's input.
Note that the array \cword{scr} is allocated.  The code will issue
error messages if more than the default \cword{nwscr}=30000 words
are needed.  Subroutine \cword{wminit} is then called.  It looks at
\index{wminit@{\ty wminit}}
the record MF=1/MT=451 for the desired material on the input GENDF tape
to obtain the group structure.  It then reads through the entire GENDF
tape for this material to set the fission flags \cword{i318} and
\cword{i618} and to count the number of temperatures that are
available.  The fission flag is used to handle cases where
both MT=18 and the partial fission representation (MT=19, 20, ...)
appear on the GENDF tape.

Subroutine \cword{resint} is called next to compute the WIMS
\index{resint@{\ty resint}}
resonance integrals from the
\hyperlink{sGROUPRhy}{GROUPR} self-shielded cross sections.
It reads through the entire GENDF tape and extracts the flux,
absorption cross section, fission cross section, and elastic
cross section versus temperature and sigma-zero for all of the
resonance groups.  It also extracts the reference-group flux
versus temperature and sigma-zero and the fission $\overline{\nu}$
value.  The latter is computed from File 6 with delayed contributions
from MT=455 in File 3.  Once all the data are in place,
\cword{resint} uses the $\overline{\nu}$ values to convert
the self-shielded fission cross sections into self-shielded
$\overline{\nu}\sigma_f$ values.

If the user has asked for a value of \cword{ntemp} that is larger
than the number of temperatures on the GENDF tape for this material,
\cword{resint} will duplicate the values from the last temperature
given into the higher temperature positions for the flux and
the absorption cross section.

The final step in \cword{wminit} is to convert the self-shielded
cross sections into resonance integrals using the method described
in Section~\ref{ssWIMSR_RI}.  These
resonance integrals are written out to a scratch
file and displayed on the output listing using subroutine \cword{rsiout}.
\index{rsiout@{\ty rsiout}}
The elastic resonance integrals are only written for WIMS-E.

Next, the main program calls \cword{xsecs} to process the cross
\index{xsecs@{\ty xsecs}}
sections.  The outermost loop is over temperature.  A distinction
has to be made between the temperature-independent matrix data,
such as (n,2n) and (n,n$'$) reactions, and the temperature-dependent
matrix data, such as thermal scattering.  While reading through
the GENDF tape, the following quantities are extracted and
stored using the allocatable arrays indicated:

\begin{list}{}{\setlength{\leftmargin}{2in}\setlength{\labelwidth}{1.7in}}
\begin{singlespace}

\item[\cword{abs1}\hfill] radiative capture (MT=102);

\item[\cword{abs2}\hfill] other absorption reactions (MT=103-150);

\item[\cword{sf0}\hfill] the fission cross section (MT=18);

\item[\cword{ab0}\hfill] also the fission cross section (MT=18);

\item[\cword{sn2n}\hfill] the (n,2n) cross section;

\item[\cword{scat}\hfill] the elastic scattering cross section (MT=2),
possibly using the value corresponding to the reference sigma-zero
value instead of infinite dilution;

\item[\cword{xi}\hfill] the log slowing down $\xi$ (MT=252);

\item[\cword{snus}\hfill] the fission yield $\overline{\nu}\sigma_f$
computed from either 3/452 or File 6 plus the delayed part
from 3/455;

\item[\cword{chi}\hfill] the fission spectrum $\chi$ computed from
either 5/455 or File 6 plus the delayed contributions from
MT=455;

\item[\cword{xs}\hfill] the temperature-dependent scattering matrix,
containing \cword{mti} and \cword{mtc}, the thermal inelastic
and elastic reactions, respectively;

\item[\cword{l1}\hfill] the smallest group number for a nonzero element
of the nonelastic part of the matrix stored in \cword{xs};

\item[\cword{l1e}\hfill] the smallest group number for a nonzero element
of the elastic part of the matrix stored in \cword{xs};

\item[\cword{l2}\hfill] the largest group number for a nonzero element
of the nonelastic part of the matrix stored in \cword{xs};

\item[\cword{l2e}\hfill] the largest group number for a nonzero element
of the elastic part of the matrix stored in \cword{xs}; and

\item[\cword{csp1}\hfill] the P$_1$ cross section for the thermal
matrix obtained by summing over the P$_1$ matrix elements for
scattering  from each group.

\end{singlespace}
\end{list}

Once all the data for a temperature have been stored in memory,
several additional operations are performed on them.  The initial
absorption cross section in \cword{ab0} contains the fission
cross section.  The final value is formed by adding the data in
\cword{abs1} and \cword{abs2}, and then subtracting the (n,2n)
cross section in \cword{sn2n}.  The final transport cross section in
\cword{xtr} is formed by adding the absorption and subtracting the
transport correction (\cword{csp1}).  The slowing-down power per unit
lethargy is computed from $\xi$, $\sigma_{scat}$, and the group
boundary energies at \cword{iegb}.

As usual, the treatment of fission is more complex.  For some
combinations of options, $\overline{\nu}$ is computed by dividing
the fission neutron production cross section by the fission cross
section, and for others, the value of $\overline{\nu}\sigma_f$ has
to be computed from $\overline{\nu}$ and $\sigma_f$.  In addition,
the fission spectrum, if requested, is normalized.

Subroutine \cword{xseco} starts with a section that writes and
prints the temperature-independent part of the WIMS data.  This
section is skipped when \cword{xseco} is called with \cword{itemp>1}.
\index{xseco@{\ty xseco}}
It first processes the temperature-independent vectors: potential
scattering, slowing-down power, transport, absorption, IR
lambda, and sometimes (n,2n).  Next, it processes the fission
vectors nu*sigf and sigf.  The temperature-independent part of
the scattering matrix includes the fast groups and the
resonance groups (this normally totals 27 groups).  It is
packed by retrieving the low (\cword{lone}) and high (\cword{ltwo})
limits for each band of nonzero elements from \cword{l1} and
\cword{l2}, which were loaded in \cword{xsecs}.  They are used to
compute the number of elements in the band and the location of the
self-scatter element in the band (always 1 for this part of the matrix,
because there is no upscatter).  They are also used to direct
how the numbers in \cword{xs} are moved into \cword{scr} with
the zeros outside of the band removed.  Note that the data are
stored as follows: location of self-scatter for group 1, number
of elements in band for group 1, the band of elements for group 1,
location of self-scatter for group 2, number of elements in group 2,
the band of elements for group 2, {\it etc.}  If the number of elements
in a band is zero, the two counts are there, but no band data are
given.  After being printed, the temperature-independent part of
the matrix is written out on a scratch file \cword{nscr2}.

The next part of \cword{xseco} is executed for \cword{itemp}=1
and all the higher temperatures.  It prints and writes the
temperature-dependent transport, and absorption cross sections
(they are defined in the thermal range only, normally groups
28-69).  Note that the absorption cross section is also written
out on scratch file \cword{nscr3}.  If available, the
temperature-dependent fission neutron production cross section and
fission cross section are also printed and written.  The
temperature-dependent scattering matrix is processed as described
above, except that the incident-neutron group range is limited to
the thermal range (normally 28-69).  Note that the location of the
self-scatter element will no longer be 1 for these data, because of
the presence of upscatter.  The packed matrix length and the packed
data are written out to scratch file \cword{nscr3}.

If the user has requested that P$_1$ scattering matrices be
constructed for the material, subroutine \cword{p1scat} is called.
\index{p1scat@{\ty p1scat}}
It uses methods similar to those described above.  The results
are printed and written onto \cword{nscr4} by \cword{p1sout}.
\index{p1sout@{\ty p1sout}}

The last step in WIMSR is to call \cword{wimout} to prepare the
\index{wimout@{\ty wimout}}
final WIMS data library.  The first card on \cword{nout} is slightly
different for WIMS-D and WIMS-E.  It is followed by lines for
the burnup data using numbers obtained from common storage.

The data needed for the material identifier card are available
in global storage.  The temperature-independent data are read from
\cword{nscr2} and written to \cword{nout}.  Similarly, the
temperature-dependent data are read from \cword{nscr3}
(although the \cword{tempr} array is obtained from global storage).
For WIMS-D, a record mark is written at this point.

The resonance data, if needed, are read from \cword{nscr1} and written
to \cword{nout}.  The format is slightly different for WIMS-D and WIMS-E.
The WIMS-D version has extra lines containing \cword{ntnp}, the product
of the number of temperatures and the number of sigma-zero values, and
it has a record mark after the resonance data block.  The WIMS-E
version has an extra section of resonance-integral data for computing
the self-shielded elastic scattering cross section.

If a fission spectrum was requested, it is written out next by
using data from the allocatable array \cword{uff}.  If a P$_1$ matrix
was requested, it is read in from \cword{nscr4} and written out
onto \cword{nout}.

This completes the entire WIMS library.  The final step takes place
in the main WIMSR program, where the normal timing and storage
usage messages are printed.


\subsection{WIMS Data File Format}
\label{ssWIMSR_fileformat}

The following section describes the WIMS data output provided
by WIMSR.  It consists of a number of logical blocks of
information written out in coded form.  The output is intended
to be used by a library maintenance code to prepare a binary
library for use by the WIMS code.
\index{WIMS format}

\noindent

\underline{Library Header \cword{(2I5)}}

\begin{list}{}{\setlength{\leftmargin}{2in}\setlength{\labelwidth}{1.7in}}
\item[\cword{NFID}\hfill] material identifier
\item[\cword{NPOS}\hfill] position to insert material on a
    large multimaterial library (given for WIMS-E only)
\end{list}

\noindent
\underline{Burnup Data \cword{(3(1PE15.8,I6))}}

\begin{list}{}{\setlength{\leftmargin}{2in}\setlength{\labelwidth}{1.7in}}
\item[\cword{(YIELD(I),IFISP(I),I=1,JCC/2}] fission yields and
   fission product flag
\end{list}

\noindent
\underline{Material Identifier Data \cword{(I6,1PE15.8,5I6)}}

\begin{list}{}{\setlength{\leftmargin}{2in}\setlength{\labelwidth}{1.7in}}
\item[\cword{IDENT}\hfill] material identifier
\item[\cword{AWR}\hfill] atomic weight ratio to neutron
\item[\cword{IZNUM}\hfill] atomic charge number $Z$ for this material
\item[\cword{IFIS}\hfill] fission and resonance flag: 0=non-fissile with
no resonance tabulation, 1=non-fissile with resonance absorption only,
2=fissile with resonance absorption only ({\it e.g.,} $^{240}$Pu),
3=fissile with resonance absorption and fission, 4=fissile with no
resonance tabulations.
\item[\cword{NTEMP}\hfill] number of temperatures
\item[\cword{NRESTB}\hfill] number of resonance tables included (0 or 1)
\item[\cword{ISOF}\hfill] fission spectrum flag (0=no, 1=yes)
\end{list}

\noindent
\underline{Temperature-Independent Vectors \cword{(1P5E15.8)}}

\begin{list}{}{\setlength{\leftmargin}{2in}\setlength{\labelwidth}{1.7in}}
\item[\cword{(SIGP(I),I=1,N2)}\hfill] $\sigma_p$ for resonance groups
\item[\cword{(XX(I),I=1,N2)}\hfill] $\xi\sigma_s/\tau$ for resonance groups
\item[\cword{(XTR(I),I=1,N1}+\cword{N2)}\hfill] transport cross section for
  fast and resonance groups
\item[\cword{(ABS(I),I=N1}+\cword{N2)}\hfill] absorption cross section for fast
    and resonance groups
\item[\cword{(DUM,I=1,N2)}\hfill] unused dummy for resonance groups
\item[\cword{(ALAM(I),I=1,N2)}\hfill] IR $\lambda$ values for resonance groups
\end{list}

\noindent
\underline{Temperature-Independent Fission (\cword{IFIS>1} only)
   \cword{(1P5E15.8)}}

\begin{list}{}{\setlength{\leftmargin}{2in}\setlength{\labelwidth}{1.7in}}
\item[\cword{(NSIGF(I),I=1,N1)}\hfill] $\overline{\nu}\sigma_f$ for fast
    and resonance groups
\item[\cword{(SIGF(I),I=1,N1)}\hfill] $\sigma_f$ for fast and resonance groups
\end{list}

\noindent
\underline{P$_0$ Matrix Length \cword{(I15)}}

\begin{list}{}{\setlength{\leftmargin}{2in}\setlength{\labelwidth}{1.7in}}
\item[\cword{NDAT}\hfill] length of P$_0$ scattering block to follow
\end{list}

\noindent
\underline{Temperature-Independent P$_0$ Matrix \cword{(1PE15.8)}}

\begin{list}{}{\setlength{\leftmargin}{2in}\setlength{\labelwidth}{1.7in}}
\item[\cword{(XS(I),I=1,NDAT)}\hfill] packed scattering data: \cword{IS} for
     group 1, \cword{NS} for group 1, \cword{NS} scattering elements
     for group 1, \cword{IS} for group 2, \cword{NS} for group 2, {\it etc.},
     through all of the fast and resonance groups (normally through
     group 27).  \cword{IS} is the position of self-scatter in the
     band of scattering elements (always 1 here), and \cword{NS} is
     the number of elements in the band.
\end{list}

\noindent
\underline{Temperature Values \cword{(1PE15.8)}}

\begin{list}{}{\setlength{\leftmargin}{2in}\setlength{\labelwidth}{1.7in}}
\item[\cword{(TEMP(I),I=1,NTEMP)}\hfill] temperatures in Kelvin
\end{list}

\begin{center}
   \it
   Repeat the following 4 blocks for\\
   each of the \cword{NTEMP} temperatures.
\end{center}

\noindent
\underline{Temperature-Dependent Transport and Absorption \cword{(1PE15.8)}}

\begin{list}{}{\setlength{\leftmargin}{2in}\setlength{\labelwidth}{1.7in}}
\item[\cword{(XTR(I),I=1,N3)}\hfill] transport cross section
    for thermal groups
\item[\cword{(ABS(I),I=1,N3)}\hfill] absorption cross section
    for thermal groups
\end{list}

\noindent
\underline{Temperature-Dependent Fission Vectors \cword{(1PE15.8)}}

\begin{list}{}{\setlength{\leftmargin}{2in}\setlength{\labelwidth}{1.7in}}
\item[\cword{(NSIGF(I),I=1,N3)}\hfill] fission neutron production
     cross section $\overline{\nu}\sigma_f$ for thermal groups
\item[\cword{(SIGF(I),I=1,N3)}\hfill] fission cross section $\sigma_f$
    for thermal groups
\end{list}

\noindent
\underline{Temperature-Dependent P$_0$ Matrix Length \cword{(I15)}}

\begin{list}{}{\setlength{\leftmargin}{2in}\setlength{\labelwidth}{1.7in}}
\item[\cword{NDAT}\hfill] length of P$_0$ scattering block
\end{list}

\noindent
\underline{Temperature-Dependent P$_0$ Matrix \cword{(1PE15.8)}}

\begin{list}{}{\setlength{\leftmargin}{2in}\setlength{\labelwidth}{1.7in}}
\item[\cword{XS(I),I=1,NDAT)}\hfill] packed scattering data: \cword{IS} for
     group N, \cword{NS} for group N, \cword{NS} scattering elements
     for group N, \cword{IS} for group N+1, \cword{NS} for group N+1, {\it etc.}
     through the last group (normally group 69).  \cword{IS} is the
     position of self-scatter in the band of scattering elements, and
    \cword{NS} is the number of elements in the band.
\end{list}

\noindent
\underline{Record Mark \cword{('   999999999999')}, WIMS-D only}
\vspace{12pt}


\noindent
\underline{Resonance Control Data}

\begin{list}{}{\setlength{\leftmargin}{2in}\setlength{\labelwidth}{1.7in}}
\item[\cword{RID}\hfill] resonance set identifier
\item[\cword{NTEMP}\hfill] number of temperatures for this resonance set
\item[\cword{NSIGZ}\hfill] number of background cross section values
\end{list}

\noindent
\underline{Absorption Resonance Data}

\begin{list}{}{\setlength{\leftmargin}{2in}\setlength{\labelwidth}{1.7in}}
\item[\cword{(TEMP(I),I=1,NTEMP)}\hfill] temperatures
\item[\cword{(SIGZ(I),I=1,NSIGZ)}\hfill] background cross section values
\item[\cword{(RESA(I),I=1,NTEMP}*\cword{NSIGZ}\hfill] absorption resonance
   integrals
\end{list}

\noindent
\underline{Nu*fission Resonance Data}

\begin{list}{}{\setlength{\leftmargin}{2in}\setlength{\labelwidth}{1.7in}}
\item[\cword{(TEMP(I),I=1,NTEMP)}\hfill] temperatures
\item[\cword{(SIGZ(I),I=1,NSIGZ)}\hfill] background cross section values
\item[\cword{(RESNF(I),I=1,NTEMP}*\cword{NSIGZ}\hfill] nu*fission resonance
   integrals
\end{list}

\noindent
\underline{Scattering Resonance Data, WIMS-E only}

\begin{list}{}{\setlength{\leftmargin}{2in}\setlength{\labelwidth}{1.7in}}
\item[\cword{(TEMP(I),I=1,NTEMP)}\hfill] temperatures
\item[\cword{(SIGZ(I),I=1,NSIGZ)}\hfill] background cross section values
\item[\cword{(RESS(I),I=1,NTEMP}*\cword{NSIGZ}\hfill] scattering resonance
   integrals
\end{list}


\noindent
\underline{Record Mark \cword{('   999999999999')}, WIMS-D only}
\vspace{12pt}


\noindent
\underline{Fission Spectrum}
\begin{list}{}{\setlength{\leftmargin}{2in}\setlength{\labelwidth}{1.7in}}
\item[\cword{(FSPECT(I),I=1,NGND)}\hfill] fission spectrum $\chi$
\end{list}


\begin{center}
     \it
     Repeat the following two blocks\\
     for each of \cword{NTEMP} temperatures.
\end{center}

\noindent
\underline{P$_1$ Matrix Length \cword{(I15)}}

\begin{list}{}{\setlength{\leftmargin}{2in}\setlength{\labelwidth}{1.7in}}
\item[\cword{NDAT}\hfill] length of P$_1$ scattering block
\end{list}

\newpage
\noindent
\underline{P$_1$ Matrix \cword{(1PE15.8)}}

\begin{list}{}{\setlength{\leftmargin}{2in}\setlength{\labelwidth}{1.7in}}
\item[\cword{XS(I),I=1,NDAT)}\hfill] packed scattering data: \cword{IS} for
     group 1, \cword{NS} for group 1, \cword{NS} scattering elements
     for group 1, \cword{IS} for group 2, \cword{NS} for group 2, {\it etc.},
     through the last group (normally group 69).  \cword{IS} is the
     position of self-scatter in the band of scattering elements, and
     \cword{NS} is the number of elements in the band.
\end{list}

\subsection{WIMSR Auxiliary Codes}
\label{ssWIMSR_aux}

The WIMSR output as described above is not directly usable by
WIMS.  Two library-maintenance codes are used at Los Alamos.
An unpublished code called FIXER is used to modify (fix up) an
existing WIMS-D library, or to create a new one, using WIMSR
output.  It processes burnup data, main data, resonance data,
and P$_1$ matrices.  In its fix-up mode, it can replace,
delete, or add a material.  another unpublished code called
WRITER is a code to read a WIMS-D library in coded form, convert
it to binary form, and list it in a user readable form.
\index{FIXER}
\index{WRITER}
\index{WIMS!WIMS-D}

\subsection{Error Messages}
\label{ssWIMSR_msg}

\begin{description}
\begin{singlespace}

\item[\cword{error in wimsr***too many time dependent isotopes}] ~\par
  See \cword{nymax}=100 in the global variables at the beginning of
  the \cword{wimsm} module.

\item[\cword{error in wminit***desired material is not on gendf tape}] ~\par
   Check whether the right input GENDF input tape was mounted.

\item[\cword{error in wminit***incorrect group structure}] ~\par
   The group structure on the input GENDF file does not agree
   with the one specified in the WIMSR input deck.

\item[\cword{message from wminit---mat xxxx mf xx has both mt18 and ...}] ~\par
   If both MT=18 and MT=19 are present in File 3 or File 6, WIMSR
   must make a choice of which to use.  For materials with partial
   fission reactions, \hyperlink{sGROUPRhy}{GROUPR} normally
   does not process the fission
   matrix from MT=18.  It is more accurate to use the sum of the
   partial fission matrices.

\item[\cword{message from wminit---mat xxxx has no mf3, mt252 ...}] ~\par
   In the absence of an input section for $\overline{\mu}$, it will be computed
   assuming isotropic CM scattering.

\item[\cword{error in resint***storage exceeded}] ~\par
   The size of the allocatable scratch array is set by the
   global parameter \cword{nwscr}=30000.

\item[\cword{error in xsecs***storage exceeded}] ~\par
   The size of the allocatable scratch array is set by the
   global parameter \cword{nwscr}=30000.

\item[\cword{message from xsecs---ref sig0 xxx not on list...}] ~\par
   In this case, the first entry will be used as the default.

\item[\cword{message from xsecs---nu-bar calculated from fission ....}] ~\par
   This is just to alert the user to this situation.

\item[\cword{message from xsecs---use only xx temps for mat xxxx}] ~\par
   The thermal inelastic sections are missing from some of the
   higher temperatures on the input GENDF tape.  This message
   tells you how many temperatures can be used correctly.  Check
   the \hyperlink{sTHERMRhy}{THERMR} and
   \hyperlink{sGROUPRhy}{GROUPR} runs if more temperatures are needed.

\item[\cword{error in xseco***scratch storage exceeded}] ~\par
   The size of the allocatable scratch array is set by the
   global parameter \cword{nwscr}=30000.

\item[\cword{error in p1scat***storage exceeded}] ~\par
   The size of the allocatable scratch array is set by the
   global parameter \cword{nwscr}=30000.

\item[\cword{error in p1scat***no p1 matrices found for mat xxxx}] ~\par
   Check the \hyperlink{sGROUPRhy}{GROUPR} run to make sure
   that P$_1$ matrices have been requested for the desired materials.

\item[\cword{error in p1scat***no temperature-dependent reactions ...}] ~\par
   Check the \hyperlink{sGROUPRhy}{GROUPR} run to make sure that
   the elastic scattering matrix was requested.

\end{singlespace}
\end{description}

\cleardoublepage

