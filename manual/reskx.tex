\section{RESKR}
\label{sRESKR}

\hypertarget{sRESKRhy}{The}
RESKR\index{RESKR|textbf} module of NJOY generates
Doppler-broadened\index{Doppler-broadening} resonance elastic scattering
kernels (RESKs) in PENDF\index{PENDF} format starting from piecewise linear
cross sections in PENDF format.  The input cross sections are recovered from
\hyperlink{sRECONRhy}{RECONR}\index{RECONR} {\sl and} from a previous
\hyperlink{sBROADRhy}{BROADR}\index{BROADR} run. The code is based on
the RESK\cite{RESK}\index{RESK} module in the ATLAS nuclear data processing
code\cite{ATLAS}\index{ATLAS} by J. Xu, T. Zu and L. Cao.

The exact Doppler broadened energy transfer kernel produces a strong
up-scattering effect in the resolved resonance energy range, above the
upper limit where the $S(\alpha,\beta)$ free gas model of the
\hyperlink{sTHERMRhy}{THERMR}\index{THERMR} module is set. The RESKR module
is used to pursue Doppler broadening of the elastic kernel for resonant
nuclides above the thermal domain considered by THERMR.

The RESKR module implements the Blackshaw-Murray\cite{BLACKSHAW} elastic
kernel to represent the effect of neutron up-scattering caused by thermal
motion of target nuclei and resonance elastic scattering on the multigroup
scattering matrices. Firstly, the resonance elastic scattering kernel (RESK)
formulations for anisotropic scattering up to any Legendre order is adopted
to represent the exact Doppler broadened energy transfer kernels. A
semi-analytical integration method is applied to perform the RESK
calculations. Combining with the RESK calculations, a linearization
algorithm is proposed to generate the RESK interpolation tables. These
interpolation tables are Legendre moments $\ell$ of the elastic scattering
kernels of the form $P_\ell(E \rightarrow E')$. They are written in the
output PENDF file as a new {\tt MF6 MT300} reaction type.

This report describes the RESKR module in NJOY 2012.0.

\subsection{The Resonance Elaslic Scattering Kernel Theory}
\label{ssRESKR_theory}

The effective cross section for a material at temperature $T$ is defined
to be that differential cross section that gives the same reaction rate
for stationary target nuclei as the real differential cross section gives
for moving nuclei. Therefore,
\begin{equation}
  v {\overline \sigma}(v,T)=
    \int d^3V \, |{\bf v}-{\bf V}|\,\sigma_{\rm s}(|{\bf v}-{\bf V}|) \,
    P{\rm s}({\bf v},{\bf V} \rightarrow {\bf v}') \,P({\bf V},T)\,\,,
\label{resk1}
\end{equation}
\noindent where
\begin{description}
\item [${\bf v}$ =] velocity of the incident neutron in LAB with $v=|{\bf v}|$
\item [${\bf V}$ =] velocity of the nucleus target in LAB
\item [${\bf v}'$ =] velocity of the scattered neutron in LAB
\item [$\sigma_{\rm s}$ =] scattering cross section in LAB
\item [$P_{\rm s}({\bf v},{\bf V} \rightarrow {\bf v}')$ =] elastic kernel
in LAB
\item [$P({\bf V},T)$ =] distribution of target velocities in LAB.
\end{description}

The elastic kernel in the CM for an isotropic collision in the CM is
written\cite{OUISLOUMEN}
\begin{equation}
P_{\rm s}({\bf u} \rightarrow {\bf u}')={\delta(u-u')\over 2u^2}
\label{resk2}
\end{equation}
\noindent where
\begin{description}
\item [${\bf u}$ =] velocity of the incident neutron in CM with $u=|{\bf u}|$
\item [${\bf u}'$ =] velocity of the scattered neutron in CM with
$u'=|{\bf u}'|$.
\end{description}

For many cases of interest, the target motion is isotropic and the
distribution of velocities can be described by the
Maxwell-Boltzmann\index{Maxwell-Boltzmann function} distribution
\begin{equation}
  P({\bf V},T)\,d^3V=
    {{\alpha^{3/2}} \over {\pi^{3/2}}} \exp (-\alpha {V}^2)\, d^3V\,\,,
\label{resk3}
\end{equation}
where $\alpha=M/(2kT)$, $k$ is Boltzmann's constant\index{Boltzmann
constant}, and $M$ is the target mass.

The general expression for the effective transfer elastic cross section
of an isotropic medium with a Maxwell-Boltzmann velocity distribution can
be obtained as\cite{OUISLOUMEN}
\begin{equation}
\sigma_{{\rm s},\ell}(E\rightarrow E')={\beta^{5/2}\over 4E} \,
\exp\left({E\over kT}\right) \int_0^\infty dt \, t \, \sigma_{\rm s}
\left({kT\over A}\, t^2\right) \exp\left({-t^2\over A}\right) \psi_\ell(t)
\label{resk4}
\end{equation}
\noindent where
\begin{description}
\item [$\ell$ =] Legendre moment ($=0$ for the isotropic component in LAB)
\item [$A$ =] mass of the nucleus in unit of neutron mass
\item [$\beta$ =] $(A+1)/A$
\item [$\sigma_{\rm s}$ =] energy-dependent 0K cross section as recovered
from input PENDF tape,
\end{description}

\noindent and where we used the integration variable $t$ as a non-dimensional
expression of the neutron velocity $u$ in CM:
\begin{equation}
t=u\, \sqrt{m(A+1)\over 2kT}
\label{resk5}
\end{equation}

\noindent so that the kinetic energy available to excite the compound nucleus
in LAB is written
\begin{equation}
E_{\rm exc}={1\over 2} {mA\over A+1} V_{\rm r}^2={kT\over A}\, t^2
\label{resk6}
\end{equation}
\noindent where $V_{\rm r}=|{\bf v}-{\bf V}|$ is the velocity of the neutron
relative to the nucleus in LAB.

The function $\psi_\ell(t)$ can be evaluated numerically as described in
Ref.~\cite{RESK}. For the case where the collision is isotropic in CM,
and for Legendre orders $\le 1$, analytical expressions are available as
\begin{eqnarray}
\nonumber \psi_0(t) \negthinspace&=& \negthinspace{\cal H}(t_+ -t)
{\cal H}(t_- -t)\left[{\rm erf}(t+\epsilon_{\rm min})
-{\rm erf}(\epsilon_{\rm max} -t)\right] \\
&+& \negthinspace {\cal H} (t - t_+)\left[{\rm erf}
(t + \epsilon_{\rm min}) - {\rm erf}(t - \epsilon_{\rm min})\right]
\label{resk7}
\end{eqnarray}

\noindent and
\begin{eqnarray}
\nonumber \psi_1(t)\negthinspace&=& \negthinspace{1\over 4 \sqrt{\pi}
\epsilon_{\rm max} \epsilon_{\rm min}} \Big[\rho \, \psi_0(t) \\
\nonumber &-& \negthinspace {\cal H}(t_+ - t){\cal H}(t - t_-)
\big( g(\epsilon_{\rm min},\epsilon_{\rm max})+
j(\epsilon_{\rm max},\epsilon_{\rm min})\big) \\
&-& \negthinspace {\cal H}(t - t_+)
\big( (g(\epsilon_{\rm min}, \epsilon_{\rm max})-
j(\epsilon_{\rm min},\epsilon_{\rm max})\big)\Big]
\label{resk8}
\end{eqnarray}

\noindent where ${\cal H}(t)$ is the Heaviside step function and where
\begin{equation}
\epsilon_{\rm max}=\sqrt{(A+1)\max(E,E')\over kT}\, ,
\label{resk9}
\end{equation}

\begin{equation}
\epsilon_{\rm min}=\sqrt{(A+1)\min(E,E')\over kT}\, ,
\label{resk10}
\end{equation}

\begin{equation}
t_\pm={\epsilon_{\rm max}\pm \epsilon_{\rm min}\over 2}\, ,
\label{resk11}
\end{equation}

\begin{equation}
\rho=\sqrt{\pi} \left[\epsilon_{\rm max}^2 +\epsilon_{\rm min}^2 -2t^2
+{1\over 2} -2(\epsilon_{\rm max}^2 -t^2)(\epsilon_{\rm min}^2-t^2)]\right]\, ,
\label{resk12}
\end{equation}

\begin{equation}
g(x,y)=\exp\left[-(t+x)^2\right] \big[ t + x - 2(y^2 - t^2)(t-x)\big]
\label{resk13}
\end{equation}

\noindent and
\begin{equation}
j(x,y) = \exp\left[-(t-x)^2\right] \big[t - x - 2(y^2 - t^2)(t+x)\big] \, .
\label{resk14}
\end{equation}

\subsection{Coding Details}
\label{ssRESKR_details}

The main entry point is subroutine \cword{reskr} exported by module
\cword{reskm}\index{modules!reskm@{\ty reskm}}. The code begins by
reading the user's input (see Section~\ref{ssRESKR_UserInp}). The coding
logic of module \cword{reskm} is similar to the one used by module
\cword{thermm}\index{modules!thermm@{\ty thermm}}.

An unassigned reaction type number \cword{MT300} in ENDF-6 format is assigned
to store and output the interpolation table of the RESK data and the
\cword{MF3 MT300} and \cword{MF6 MT300} reaction types are defined. The
incident energy grid of reaction \cword{MF3 MT300} is a subset of the energy grid
in \cword{MF3 MT2} selected between lower (\cword{elo}) and upper (\cword{ehi})
incident energy boundaries for the RESK calculation. The incident energy grid
of reaction \cword{MF6 MT300} is a coarser grid set to reduce computing cost.

At each reconstructed incident energy, the moments of energy transfer
kernels for the different orders are linearized simultaneously on a
single unionized grid by the conventional interval-halving techniques
of Cullen\cite{SIGMA1}. It ensures that all orders of the moments are
reconstructed smoothly. Meanwhile, the moments of energy transfer kernels
for the different orders are interpolated simultaneously once the
interpolation interval is found.

A data structure in ENDF-6 format is defined as described in
Table~\ref{reskr_tab1} where \cword{HEAD}, \cword{TAB2}, \cword{TAB1} and
\cword{LIST} are the standard types of records; \cword{ZA} and \cword{AWR}
are the standard material charge and mass parameters; \cword{NL} is the
maximum Legendre order number of this table; \cword{T} is the absolute
temperature; \cword{E1} is the primary energy; \cword{EP} is the secondary
energy; \cword{P0} indicates the 0th moment of energy transfer kernel and
\cword{P1} represents the 1st moment of energy transfer kernel. The
\cword{TAB1} structure with embedded \cword{LIST} structures is repeated
for all \cword{NE} incident energies. Each \cword{TAB1} structure contains
\cword{NL}$-1$ embedded \cword{LIST} structures containing kernel values for
$P_1$, $P_2$ and higher Legendre moments.

\begin{table}[t]
\caption[MF6 MT300 reaction type.]{MF6 MT300 reaction type.}
\begin{center}
\begin{BVerbatim}
[MAT, 6, MT / ZA, AWR, NL, 0, 0, 0 ] HEAD
[MAT, 6, MT / T, 0.0, 0, 0, NR, NE / (INT) ] TAB2
[MAT, 6, MT / 0.0, E1, 0, 0, NRP, NEP / EP / P0(E1->EP) ] TAB1
[MAT, 6, MT / 0.0, 0.0, 0, 0, NW, 0 / P1(E1->EP) ] LIST
Repeat the LIST structure for all NL-1 orders
Repeat the TAB1 structure for all NE incident energies
\end{BVerbatim}
\label{reskr_tab1}
\end{center}
\end{table}

After the processing, the data will be output into derived files which are the
point-ENDF (PENDF) files. The reusable PENDF files can be used for generating
the different multi-group cross sections and scattering matrices faced with
the different requirements of dilutions and energy group structures.

\subsection{User Input}
\label{ssRESKR_UserInp}

The following description of the user input is reproduced from
the comment cards at the beginning of the RESKR module.
\index{RESKR!RESKR input}
\index{input!RESKR}

\small
\begin{ccode}

    !---input specifications (free format)---------------------------
    !
    ! card 1
    !    nin1     input pendf tape containing 0K data
    !    nin2     input pendf tape containing Doppler broadened data
    !    nout     output pendf tape
    ! card 2
    !    mat1     material to broadened
    !    ntemp2   number of final temperatures (default=1)
    ! card 3
    !    elo      lower incident energy boundary for the RESK calculation
    !    ehi      upper incident energy boundary for the RESK calculation
    !    lord     maximum legendre order
    ! card 4
    !    temp2    final temperatures (deg Kelvin)
    ! card 5
    !    mat1     next MAT number to be processed with these
    !             parameters. Terminate with mat1=0.
    !
    !-------------------------------------------------------------------

\end{ccode}
\normalsize

Material numbers (\cword{matb}) are simply the isotopic {\tt ZA} parameter
for ENDF evaluations; they are equal to $100{*}Z$ and two last digits function
of $A$ for ENDF$\ge$6 formatted files.

The following sample run prepares a PENDF tape for uranium-238.
The numbers on the left are for reference; they are not part of the input.

\small
\begin{ccode}

  1.   reconr
  2.   20 -21
  3.   reconr
  4.   -21 -22
  5.   'pendf tape from /tmp/u238_b7r1'/
  6.   9237 1/
  7.   0.001  0.  0.005/
  8.   'U238 from /tmp/u238_b7r1 at Wed May 26 11:03:52 2021' /
  9.   0/
 10.   moder
 11.   -22 31
 12.   broadr
 13.   -21 -22 -23
 14.   9237 2/
 15.   0.001/
 16.   900.0 2000.0/
 17.   0/
 18.   moder
 19.   -23 32
 20.   reskr
 21.   31 32 33
 22.   9237 2/
 23.   4.0 100.0 1
 24.   900.0 2000.0/
 25.   0/

\end{ccode}
\normalsize

Two PENDF tapes are produced before calling RESKR: the first, on unit 31, for
a nucleus at 0K and the second, on unit 32, containing broadened data. Next,
module RESKR is called for broadening the elastic scattering kernel between
4 and 100 eV and 2 Legendre orders. The output PENDF tape is recovered on unit
33.

\subsection{I/O Units}
\label{ssRESKR_IO}

There are no scratch files used in RESKR.

\cleardoublepage
