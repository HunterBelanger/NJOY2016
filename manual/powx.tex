\section{POWR}
\label{sPOWR}

\hypertarget{sPOWRhy}{The}
POWR module is used to prepare libraries for the Electric
\index{POWR|textbf}
Power Research Institute (EPRI)\index{EPRI} reactor analysis codes
EPRI-CELL and EPRI-CPM.\footnote{EPRI-CELL and EPRI-CPM are
proprietary products of the Electric Power Research Institute,
3420 Hillview Avenue, Palo Alto, CA 94304.  For more information,
please contact the owners.}\index{Electric Power Research Institute|see{EPRI}}
\index{EPRI!EPRI-CELL}\index{EPRI!EPRI-CPM}
These codes were originally developed to provide an alternative
to the reactor manufacturers'
computer codes for calculating reactor core performance as
required for operating and reloading power reactors run by
US electric utility companies.  Because these codes require
up-to-date accurate cross sections that were not controlled
by the reactor manufacturers, EPRI contracted with the Los
Alamos National Laboratory to generate new libraries based on
the US-standard Evaluated Nuclear Data Files (ENDF).  With this
funding, we were able to develop the
\hyperlink{sTHERMRhy}{THERMR} module and the
associated thermal multigroup methods in
\hyperlink{sGROUPRhy}{GROUPR}.  We also
developed this POWR module to format cross section data in
GENDF format for use in CELL or CPM.  In addition, it was necessary
to make a number of modifications to EPRI-CELL to make it perform
well with unadjusted cross sections.  The results of all this
work were reported in 1984\cite{powr}.
\index{THERMR}
\index{GROUPR}

This module has not been used at Los Alamos since 1984,
although we have had scattered reports of use elsewhere.  A list of
the input instructions, without further comment, follows.


\subsection{Input Instructions}
\label{ssPOWR_inp}

As an aid to discussions of the user input to POWR, the input
instructions that appear as comment cards at the beginning of the
current version of this module are listed below.  Since code changes
are possible, albeit unlikely for this module, it is always advisable
to consult the comment-card instructions contained in the version of
the code actually being used before proceeding with an actual calculation.
\index{POWR!POWR input}
\index{input!POWR}

\small
\begin{ccode}

   !---input specifications (free format)---------------------------
   !
   ! card 1
   !    ngendf  unit for input gout tape
   !    nout    unit for output tape
   ! card 2
   !    lib     library option (1=fast, 2=thermal, 3=cpm)
   !    iprint  print option (0=minimum, 1=maximum)
   !            (default=0)
   !    iclaps  group collapsing option (0=collapse from 185 group
   !            to desired group structure, 1=no collapse)
   !            (default=0)
   !
   !---for lib=1----------------------------------------------------
   !
   ! card 3
   !    matd    material to be processed
   !            if matd lt 0, read-in absorption data only for
   !            this material with mat=abs(matd) directly from
   !            input deck (see card 6)
   !   following three parameters irrelevant for matd lt 0
   !    rtemp   reference temperature (degrees kelvin)
   !            (default=300  k)
   !    iff     f-factor option
   !            (0/1=do not calculate f-factors/calculate if found)
   !            (default=1)
   !    nsgz    no. of sigma zeroes to process for this material
   !            (default=0=all found on input tape)
   !    izref   ref. sigzero for elastic matrix (default=1)
   ! cards 4 and 5 for normal run only (matd gt 0)
   ! card 4
   !    word    description of nuclide (up to 16 characters,
   !            delimited with *, ended with /)  (default=blank)
   ! card 5
   !    fsn     title of fission spectrum (up to 40 characters,
   !            delimited with *, ended with /0  (default=blank)
   !             delimited with *, ended with /)  (default=blank)
   ! card 6 for reading in absorption data only
   !    abs     ngnd absorption values (default values=0)
   ! repeat cards 3 through 6 for each material desired.
   ! terminate with matd=0/ (i.e., a 0/ card).
   !
   !---for lib=2----------------------------------------------------
   !
   ! card 3
   !    matd    material to be processed
   !    idtemp  temperature id (default=300  k)
   !    name    hollerith name of isotope (up to 10 characters,
   !            delimited with *, ended with /)  (default=blank)
   ! card 4     default for all values=0.
   !    itrc    transport correction option (0 no, 1 yes)
   !    mti     thermal inelastic mt
   !    mtc     thermal elastic mt
   ! card 5     default for all values=0.
   !    xi
   !    alpha
   !    mubar
   !    nu
   !    kappa fission
   !    kappa capture
   !    lambda
   !    sigma s   if 0, set to scattering cross section at group 35
   ! repeat cards 3 thru 5 for each material and temperature desired
   ! (maximum number of temperatures allowed is 7.)
   ! terminate with matd=0/ (i.e., a 0/ card).
   !
   !---for lib=3----------------------------------------------------
   !
   ! card 3
   !    nlib    number of library.
   !    idat    date library is written (i format).
   !    newmat  number of materials to be added.
   !    iopt    add option (0=mats will be read in,
   !                        1=use all mats found on ngendf).
   !    mode    0/1/2=replace isotope(2) in cpmlib/
   !                   add/create a new library (default=0)
   !    if5     file5 (burnup data) option
   !            0/1/2=do not process file5 burnup data/
   !                   process burnup data along with rest of data/
   !                   process burnup data only (default=0)
   !            (default=0)
   !    if4     file4 (cross section data) option
   !            0/1=do not process/process
   !            (default=1)
   ! card 4 for iopt=0 only
   !    mat     endf mat number of all desired materials.
   !            for materials not on gendf tape, use ident for mat.
   !            if mat lt 0, add 100 to output ident
   !            (for second isomer of an isotope)
   ! card 5
   !    nina    nina indicator.
   !            0/1/2/3=normal/
   !                  no file2 data, calculate absorption in file4/
   !                  no file2 data, read in absorption in file4/
   !                  read in all file2 and file4 data.
   !    ntemp   no. of temperatures to process for this material
   !            (default=0=all found on input tape)
   !    nsigz   no. of sigma zeroes to process for this material
   !            (default=0=all found on input tape)
   !    sgref   reference sigma zero
   !    following 2 parameters are for nina=0 or nina=3.
   !    ires    resonance absorber indicator (0/1=no/yes)
   !    sigp    potential cross section from endf.
   !    following 5 parameters are for ntapea=0 only
   !    mti     thermal inelastic mt
   !    mtc     thermal elastic mt
   !    ip1opt  0/1=calculate p1 matrices/
   !                correct p0 scattering matrix ingroups.
   !  ******if a p1 matrix is calculated for one of the isotopes
   !        having a p1 matrix on the old library, file 6 on the
   !        new library will be completely replaced.******
   !    inorf   0/1=include resonance fission if found/
   !                do not include
   !    following two parameters for mode=0 only
   !    pos     position of this isotope in cpmlib
   !    posr    (for ires=1) position of this isotope in resonance
   !            tabulation in cpmlib
   !  repeat card 5 for each nuclide.
   ! following three cards are for if5 gt 0 only
   ! card 6
   !    ntis    no. time-dependent isotopes
   !    nfis    no. fissionable burnup isotopes
   ! card 7
   !    identb  ident of each of the nfis isotopes
   ! card 8
   !    identa  ident of time-dependent isotope
   !    decay   decay constant (default=0.)
   !    yield   nfis yields (default=0.)
   !  repeat card 8 for each of the ntis isotopes.
   ! card 9 for if5=2 only
   !    aw      atomic weight
   !    indfis  fission indicator
   !    ntemp   no. temperatures on old library
   !  repeat card 9 for each of the ntis isotopes.
   ! card 10
   !    lambda  resonance group goldstein lambdas
   !  ******remember that the 69-group structure has 13 resonance
   !        groups while the collapsed 185-group structure has 15.
   !        use a slash at end of each line of card 10 input.******
   !  repeat card 10 for each nuclide having nina=0, nina=3, or
   !                     ires=1.
   ! cards 11 and 11a for nuclides having nina=3 only.
   ! card 11
   !    resnu    nrg nus values to go with the lambda values
   ! card 11a
   !    tot      nrg total xsec values to go with the lambda values
   !  read cards 11 and 11a for each nuclide having nina=3.
   ! cards 12 for nina gt 2 only
   !    aw      atomic weight
   !    temp    temperature
   !    fpa     ngnd absorption values (default=0.)
   ! cards 12a, 12b, 12c for nuclides having nina=3 only.
   ! card 12a
   !    nus      ngnd nus values
   !    fis      ngnd fission values
   !    xtr      ngnd transport values
   ! card 12b
   !    ia       group.  0 means no scattering from this group
   !    l1       lowest group to which scattering occurs
   !    l2       highest group to which scattering occurs
   ! card 12c  for ia gt 0 only
   !    scat     l2-l1+1 scattering values
   !  repeat card 12b and 12c for each group
   ! repeat cards 12 for each of the nina gt 2 nuclides
   !
   !--------------------------------------------------------------------

\end{ccode}
\normalsize

\cleardoublepage

